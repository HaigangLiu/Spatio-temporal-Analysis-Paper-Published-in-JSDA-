\nonumber
\subsection{Sparse Matrix}\label{subsec:sparse}

A \emph{sparse} matrix is a matrix where most elements are zero.
By contrast, a matrix is considered \emph{dense} if most elements are nonzero.
A measure to quantify the sparsity of a matrix is the number of zero-valued elements divided by the total number of elements.
As a rule of thumb, a matrix is considered sparse when its sparsity is greater than 0.5. The covariance matrix in the aforementioned CAR model is largely based on our adjacency model, and has a sparsity of 0.86.\\

Once a sparse matrix is recognized, one can use specialized algorithms and data structures to accelerate computation.
This is because memory and computing power are wasted on the zeroes if we employ a standard dense-matrix algorithm.
Specifically, a dense matrix is typically stored as a two-dimensional array, and each entry in the array represents an element $ a_{ij}$ of the matrix.
One can access any element by specifying the row index $i$ and the column index $j$.
In contrast, in a typically sparse matrix representation, only the nonzero entries are stored and thus memory use can be reduced substantially.
As a tradeoff, retrieving individual elements becomes more complex in a sparse matrix.\\

In practice, there are several representations of a sparse matrix.
While some types stand out for their efficient modification, such as DOK (Dictionary of Keys) and COO (Coordinate List), others, e.g., Compressed Sparse Row (CSR), support fast matrix operations.
CSR suits our needs better since evaluating a multivariate normal distribution involves matrix multiplication, and thus is implemented as part of our model. \\

The compressed sparse row (CSR) represents a matrix by three one-dimensional arrays: the nonzero values, the row indices, and the column indices.
Note that the row indices are not defined in a straightforward manner.
An example is given as follows to demonstrate how a CSR representation is implemented.

\begin{align*}
\left(
\begin{matrix}
  0 & 0 & 0 & 0 \\
  5 & 8 & 0 & 0 \\
  0 & 0 & 3 & 0 \\
  0 & 1 & 0 & 0
\end{matrix}
\right)
\end{align*}

\noindent The three vectors to represent the example sparse matrix is 
\begin{verbatim}
A  = [5, 8, 3, 6]   IA = [0, 0, 2, 3, 4]   JA = [0, 1, 2, 1].
\end{verbatim} 

The array \texttt{A} is the nonzero values, whose column indices are stored in  \texttt{JA}.
For instance, 3 is in the third column and thus the third element in \texttt{JA}  is 2, which stands for the third column since a zero-based index is used.
On the other hand, \texttt{IA}  contains the row information and is defined recursively, where  \texttt{IA[0] = 0}  and  \texttt{IA[i] = IA[i-1] + k}.
Note that \texttt{k} is  number of nonzero elements on the $i$th row in the original matrix.
According to this definition, the length of \texttt{IA} is $m + 1$ when the matrix has $ m $ columns, and the last element in \texttt{IA} is always the number of nonzero values.\\

The sparse matrix stored in CSR is efficient in matrix-vector multiplication due to the structure of \texttt{IA} and \texttt{JA}.
For instance, multiplying \texttt{[5, 8, 0,0]} by another vector, say \texttt{[1, 0, 9, 9]}, requires only retrieving nonzero values at location 0 and 1 from the second row.
The location information is conveniently stored in \texttt{JA}, and the length of nonzero values can be found in \texttt{IA}.
 Since we only need to compute the dot product of \texttt{[5, 8]} and \texttt{[1, 0]}, the computation is reduced by half.
  In practice, we observe a six-to-ten times boost in sampler performance by switching from a dense matrix implementation, since the adjacency matrix has greater sparsity.


\begin{thebibliography}{9}
\bibitem{7}
Banerjee, S., Carlin, B. P.,  Gelfand, A.\ E.\ (2014).\ Hierarchical modeling and analysis for spatial data.
CRC Press, Boca Raton.

\bibitem{runoff}
Betson, R. P.\ (1964).\ What is watershed runoff?
Journal of Geophysical Research, 69(8):1541-1552.

\bibitem{bonnin}
Bonnin, G. M., Martin, D., Lin, B., Parzybok, T., Yekta, M., \& Riley, D.\ (2006).\
Precipitation-frequency atlas of the United States.
NOAA atlas, 14(2): 1--65.

\bibitem{random-forest}
\hl{Breiman} L.\ (2001).\ Random forests.
Machine Learning 45(1): 5--32.
\url{https://doi.org/10.1023/A:1010933404324}

\bibitem{clark}
Clark MP et al.\ (2015).\ A unified approach for process-based hydrologic modeling:
1.\ Modeling concept.
Water Resource Research 51: 2498--2514. \url{https://doi.org/10.1002/2015WR017198}

\bibitem{new11}
Ciach, G. J.,  Krajewski, W. F.\ (2006).\ Analysis and modeling of spatial correlation structure in small-scale rainfall in central Oklahoma.
Advances in Water Resources, 29(10), 1450--1463.

\bibitem{2} 
Cressie, N. (1993).\ Statistics for spatial data.
John Wiley \& Sons, New York.

\bibitem{flood2}
Daly, C., Neilson, R. P.,  Phillips, D. L.\ (1994).\ A statistical-topographic model for mapping climatological precipitation over mountainous terrain.
Journal of Applied Meteorology, 33(2): 140--158.

\bibitem{flood1}
Daly, C., Taylor, G. H., Gibson, W. P., Parzybok, T. W., Johnson, G. L., Pasteris, P. A.\ (2000).\
High-quality spatial climate data sets for the United States and beyond.
Transactions of the ASAE, 43(6): 1957.

\bibitem{new14}
Deidda, R.\ (2000).\ Rainfall downscaling in a space-time multifractal framework.
Water Resources Research, 36(7): 1779--1794

\bibitem{new23}
Delhomme, J. P.\ (1978).\ Kriging in the hydrosciences.
Advances in Water Resources, 1(5): 251--266.

\bibitem{new22?}
Delfiner, P., Delhomme, J. P.\ (1975).\ Optimum interpolation by kriging.
Ecole Nationale Sup{\'e}rieure des Mines, Paris.

\bibitem{new9}
Diggle, P. J., Tawn, J. A., Moyeed, R. A.\ (1998).\  Model-based geostatistics.
Journal of the Royal Statistical Society: Series C (Applied Statistics), 47(3): 2299--350

\bibitem{new26}
Dumitrescu, A., Birsan, M. V., Manea, A.\ (2016).\ Spatio-temporal interpolation of sub-daily (6 h) precipitation over Romania for the period 1975--2010.
International Journal of Climatology, 36(3): 1331-1343.

\bibitem{strm}
Farr, T. G., Rosen, P. A., Caro, E., Crippen, R., Duren, R., Hensley,  Kobrick, M. et al.\ (2007).\ The shuttle radar topography mission.
Reviews of Geophysics, 45(2).

\bibitem{new12}
Ferraris, L., Gabellani, S., Rebora, N.,  Provenzale, A.\ (2003).\
A comparison of stochastic models for spatial rainfall downscaling.
Water Resources Research, 39(12).

\bibitem{fred}
Frederick, R.H.\ and Miller, J.F.\ (1979).\ Short duration rainfall frequency relations for California.
Third Conference on Hydrometeorology, Bogata, Columbia.
American Meteorological Society, 667--73

\bibitem{greedy_function}
\hl{Friedman}, J. H.\ (2001).\ Greedy function approximation: A gradient boosting machine.
The Annals of Statistics 29(5): 1189--1232. \url{https://doi.org/10.1214/aos/1013203451}

\bibitem{new15}
Georgakakos, K. P., Kavvas, M. L.\ (1987).\ Precipitation analysis, modeling, and prediction in hydrology.
Reviews of Geophysics 25(2): 163--178.

\bibitem{new8}
Hastings, W. K.\ (1970).\ Monte Carlo sampling methods using Markov chains and their applications.
Biometrika, 57(1): 97--109.

\bibitem{koutsoyiannis2}
Koutsoyiannis D.\ (2011).\ Hurst-Kolmogorov dynamics and uncertainty.
J Am Water Resour Assoc, 47(3): 481--95.
\url{http://dx.doi.org/10.1111/j.1752- 1688.2011.00543.x.}

\bibitem{koutsoyiannis3}
Koutsoyiannis D, Montanari A.\ (2014).\ Negligent killing of scientific concepts: the stationarity case.
Hydrol Sci J. \url{http://dx.doi.org/10.1080/ 02626667.2014.959959.}

\bibitem{new13}
Kumar, P., Foufoula-Georgiou, E.\ (1994).\ Characterizing multiscale variability of zero intermittency in spatial rainfall.
Journal of Applied Meteorology, 33(12): 1516--1525.

\bibitem{new25}
Ly, S., Charles, C., Degre, A.\ (2011).\ Geostatistical interpolation of daily rainfall at catchment scale:
The use of several variogram models in the Ourthe and Ambleve catchments, Belgium.
Hydrology and Earth System Sciences, 15(7): 2259--2274.

\bibitem{14}
Matheron, G.\ (1963).\ Principles of geostatistics.
Economic geology, 58(8): 1246--1266.

\bibitem{new17}
Isaaks, H.\ E., Srivastava, R.\ M.\ (1989).\ Applied geostatistics.
Oxford University Press, New York.

\bibitem{new7}
Metropolis, N., Rosenbluth, A. W., Rosenbluth, M. N., Teller, A. H., Teller, E.\ (1953).\
Equation of state calculations by fast computing machines.
The Journal of Chemical Physics, 21(6):1087-1092.

\bibitem{10} 
National Oceanic and Atmosphere Administration, U.S.\ Department of Commerce.\ (2015).\
Service assessment: The historic South Carolina floods of October 1--5, 2015.
Accessed at \path{www.weather.gov/media/publications/assessments/SCFlooding_072216_Signed_Final.pdf} on December 4, 2017.

\bibitem{phillips}
Phillips, R. C., Samadi, S., Meadows, M.E.\ (2018).\ How extreme was the October 2015 flood in the Carolinas?
An assessment of flood frequency analysis and distribution tails.
Journal of Hydrology, DOI: https://doi.org/10.1016/j.jhydrol.2018.05.035

\bibitem{samadi2}
Samadi S., Tufford D., Carbone, G.\ (2018).\
Estimating hydrologic model uncertainty in the presence of complex residual error structures.
Stochastic Environmental Research and Risk Assessment, 32(5): 1259--1281. \url{DOI: 10.1007/s00477-017-1489-6.}

\bibitem{samadi}
Samadi, S. Z.,  M. E. Meadows.\ (2017).\
The transferability of terrestrial water balance components under uncertainty and nonstationarity:
A case study of the coastal plain watershed in the southeastern USA\@.
River research and applications, 33, no.\ 5:796-808.

\bibitem{reviewer2}
Serinaldi, F., Kilsby, C.G.\ (2014).\
Rainfall extremes: Toward reconciliation after the battle of distributions.
Water Resources Research, 50: 336--352.

\bibitem{seaber}
Serinaldi, F.,  Kilsby C. G.\ (2015).\
Stationarity is undead: Uncertainty dominates the distribution of extremes.
Adv.\ Water Resour 77: 17--36, \url{doi:10.1016/j.advwatres.2014.12.013.}

\bibitem{huc}
Seaber, P.\ R., Kapinos, F.\ P.,  Knapp, G. L.\ (1987).\ Hydrologic unit maps.
Accessed at \url{https://pubs.usgs.gov/wsp/wsp2294/html/pdf.html}

\bibitem{new24}
Sharon, D.\ (1972).\ Spatial analysis of rainfall data from dense networks.
 Hydrological Sciences Journal 17(3): 291--300.

\bibitem{randomforest_usage}
Sihag, P., Mohsenzadeh S.,  Angelaki, A.\ (2019).\ Random forest, M5P and regression analysis to estimate the field unsaturated hydraulic conductivity.
Applied Water Science, 9(5): 129.
\url{https://doi.org/10.1007/s13201-019-1007-8}

\bibitem{distribution_1}
Smith, J.\ A., Villarini G., Baeck, M.\ L.\ (2011).\ Mixture Distributions and the Hydroclimatology of Extreme Rainfall and Flooding in the Eastern United States.
Journal of Hydrometeorology, 12, 294--309.
\url{https://doi.org/10.1175/2010JHM1242.1.}

\bibitem{new20}
Thiessen, A.\ H.\ (1911).\ Precipitation averages for large areas.
Monthly Weather Review, 39(7): 1082--1084.

\bibitem{randomforest_usage_3}
Tyralis, H., Papacharalampous, G., Langousis, A.\ (2019).\
A brief review of random forests for water scientists and practitioners and their recent history in water resources.
Water, 11(5): 910.

\bibitem{loo_waic}
Vehtari, A., Gelman, A.,  Gabry, J.\ (2017).\
Practical Bayesian model evaluation using leave-one-out cross-validation and WAIC\@.
Statistics and Computing, 27(5): 1413--1432.

\bibitem{reviewer}
Villarini, G.\ et al.\ (2009) .\ Flood frequency analysis for nonstationary annual peak records in an urban drainage basin.
Advances in Water Resources 32: 1255--1266.

\bibitem{randomforest_usage_2}
Walsh, E.S., Kreakie, B.J., Cantwell, M.G.,  Nacci, D.\ (2017).\
A Random Forest approach to predict the spatial distribution of sediment pollution in an estuarine system.
PloS one, 12(7): e0179473.

\bibitem{ggplot}
Wickham, Hadley.\ (2016).\ ggplot2: Elegant graphics for data analysis.
New York, Springer.

\end{thebibliography}