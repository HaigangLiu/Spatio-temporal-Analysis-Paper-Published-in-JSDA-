
We have presented a spatiotemporal model for gage height in South Carolina from 2011 to 2015, a period including one of the most destructive storms in state history.
Our model accounts for the heavy-tailed pattern of the response variable and allows us to determine several covariates that affect the gage height and to interpret their effects.
In particular, due to the effect of interactions, a stronger association between precipitation and flooding can be observed during summer compared to other times of the year.
If reliable precipitation forecasts are available, our model could be used for forecasting realtime and future flood events, potentially aiding early warning systems and emergency management.\\

In addition, we developed a Python library to streamline the data preprocessing steps.
Data scraping,  cleaning, aggregating and transforming steps can be done by simple function calls.
We demonstrate several reusable modules we have developed  by providing some basic examples in the Github repository of our package.
Our hope is that such tools will enable straightforward employment of similar spatio-temporal models for flood data in the future.
