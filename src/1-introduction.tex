
During October 2--5, 2015, an extraordinary rainfall event took place in the Carolinas, many parts of which observed 500-year-event levels of precipitation.
Accumulation of rainfall amount reached 24.23 inches \hl{(61.57 centimeters)} near Boone Hall (Mount Pleasant, Charleston County) by 11:00 a.m.\ Eastern Time on October 4, 2015.
The rainfall peaked on October 4 with a 24-hour total of 16.69 inches \hl{(42.39 centimeters)} of precipitation;
and the total 48-hour precipitation during October 3--4 was more than 20 inches \hl{(51 centimeters)}.
The likelihood of the rainfall amounts ranged from anywhere between a 1-in-250-year event to a 1-in-1000-year event in the study region with some places such as Columbia and Lexington, SC receiving more than 17 inches \hl{(43 centimeters)} of rain over a four-day period (Philips et al., 2018).
Columbia, the capital of South Carolina, broke its all-time wettest 1-day, 2-day, and 3-day periods on record (e.g., Bonnin et al., 2006).
The rainfall in Columbia far exceeded the two values of National Oceanic and Atmospheric Administration (NOAA) calculated 1,000-year events of 12.8 inches \hl{(32.5 centimeters)} and 14.1 inches \hl{(35.8 centimeters)}, respectively (NOAA Atlas 14 volume 2;
see Frederick et al., 1979).
Charleston International Airport observed a record 24-hour rainfall of 11.5 inches \hl{(29.2 centimeters)} on October 3 (Santorelli, Oct.\ 4, 2015).
Some areas experienced more than 20 inches \hl{(51 centimeters)} of rainfall over the five-day period. \\

Flooding from this event resulted in 19 fatalities, according to the South Carolina Emergency Management Department, and South Carolina state officials reported damage losses of \$1.492 billion (NOAA, U.S.\ Department of Commerce, 2015).
The heavy rainfall and floods, combined with aging and inadequate drainage infrastructure, resulted in the failure of many dams and flooding of many roads, bridges, and conveyance facilities, thereby causing extremely dangerous and life-threatening situations. \\

The rainfall event was generated by the movement of very moist air over a stalled frontal boundary near the coast.
The clockwise circulation around a stalled upper level low over southern Georgia directed a narrow plume of tropical moisture northward and then westward across the Carolinas over the course of four days.
A low-pressure system off the U.S.\ southeast coast,  as well as tropical moisture related to Hurricane Joaquin (a category 4 hurricane) was the underlying meteorological cause of the record rainfall over South Carolina during October 1--5, 2015 (NOAA, U.S.\ Department of Commerce, 2015). \\

In this article, we use geostatistical analysis to  investigate the stochastic relationship and  the dynamics between rainfall and flooding.
Spatial statistics methods have been frequently used in applied statistics as well as water resources engineering.
 The work of Thiessen (1911) was the first attempt at using interpolation methods in hydrology.
Sharon (1972) used an average of the observations from a number of rain gages to obtain estimates of the areal rainfall.
 Soon after,  Delfmer and Delhomme (1975) and Delhomme (1978) applied various geostatistical methods such as variograms and kriging methods in modeling rainfall.
 The work of Troutman (1983), Tabios and Salas (1985), Georgakakos and Kavvas (1987), Isaaks and Srivastava (1989), Kumar and Foufoula-Georgiou (1994), Deidda (2000), Ferraris et al.\ (2003), Ciach and Krajewski (2006), Berne et al.\ (2009),  Ly et al.\ (2011), \hl{Villarini} et al.\ (2009), \hl{Serinaldi} and Kilsby (2014), and Dumitrescu et al.\ (2016) further advanced the application of geostatistical methods in rainfall and flood analysis.
  The theoretical basis of the geostatistical approach was strengthened using Bayesian inference via the Markov Chain Monte Carlo (MCMC) algorithm introduced by Metropolis et al.\ (1953).
  MCMC was subsequently adapted by Hastings (1970) for statistical problems and further applied by Diggle et al.\ (1998) in geostatistical studies.\\

This article is arranged as follows: In Section~\ref{sec:source}, we provide an overview of our use of data munging to obtain the precipitation and gage height data, since the scraping, cleaning, aggregating and transforming of data constitute a major part of our study.
 Section~\ref{sec:watershed} discusses the binary adjacency matrix, which is pivotal to the conditional autoregressive model since it accounts for the spatial correlation based on watershed information.
In Section~\ref{sec:model}, our model fitting approach and results are detailed, including a remedy for some noted heavy-tailed error behavior.
Lastly, we compare our results using the conditional autoregressive model with results using other popular models such as random forest (RF), based on metrics like mean square error.